\documentclass{article}


\usepackage{amsmath}
\usepackage{bbm}
\usepackage{geometry}
 \geometry{
 a4paper,
 total={170mm,257mm},
 left=20mm,
 top=20mm,
 }



\title{Delayed Sequential Importance Sampling}

\begin{document}
\maketitle

\noindent Consider the standard sequential importance sampling with resampling (SIR) algorithm:\\

\textbf{Algorithm 1} Sequential Importance Resampling
\begin{itemize}
\item Sample $X_t^{(i)} \sim q( \cdot | X_{t-1}^{(i)})$
\item $w(X_{1:t}^{(i)}) \propto W(X_{1:t-1}^{(i)}) \cdot g(X_t^{(i)}, y_t) \cdot
  f(X_t^{(i)} | X_{t-1}^{(i)}) / q(X_t^{(i)} | X_{t-1}^{(i)})$
\item $W(X_{1:t}^{(i)}) = w(X_{1:t}^{(i)}) / \sum_{j=1}^N w(X_{1:t}^{(j)})$
\item If $ESS < N/2$:\\
resample proportional to $\{W(X_{1:t}^{(i)})\}_{i=1,...,N}$\\
$\forall i$ set $W(X_{1:t}^{(i)}) = 1/N$
\end{itemize}

\vspace{1cm}

\noindent Target distribution: $p(X_{1:t},y_{1:t}) = g(X_1, y_1) \cdot \prod_{j=2}^t [f(X_j|X_{j-1}) \cdot g(X_j, y_j)]$\\\\
\noindent Proposal distribution: $q(X_{1:t})= q(X_1) \cdot \prod_{j=2}^t q(X_j|X_{j-1})$\\\\
\vspace{1cm}

\noindent We have chosen a broad proposal distribution $q( \cdot | X_{t-1}^{(i)} )$ which produces samples with 
relatively low $ f(X_t^{(i)} | X_{t-1}^{(i)}) $, which may have high $ g(X_t^{(i)}, y_{t:t+\delta}) $.
The SIR algorithm above would quickly kill off these particles with low $ f(X_t^{(i)} | X_{t-1}^{(i)}) $
before we collected the support from $y_{t:t+\delta}$. This is primarily due to the low informativity of
our data leading to a large $\delta$ meaning we are likely to meet the
$ESS < N/2$ condition well before iteration $t+\delta$. Our solution to this problem
is to delay the presumed penalty of $ f(X_t^{(i)} | X_{t-1}^{(i)}) $ until iteration
$t+\delta$. We will still track the weights as before, now denoted $w_T(X_{1:t}^{(i)})$.
Applying these weights to the particles gives a sample from the target distribution. 
We will also track weights $w_D(X_{1:t}^{(i)})$ and all resampling decisions will be made
according to these delayed weights. Applying these weights to the particles
gives a sample from an intermediate resampling distribution.\\

\vspace{1cm}
\noindent Resampling distribution: $r(X_{1:t})= \prod_{j=1}^{t} [ g(X_j, y_j) ]
                                   \cdot \prod_{j=2}^{t-\delta} [ f(X_j | X_{j-1} ) ]
                                   \cdot \prod_{j=t-\delta+1}^{t} [ q(X_j | X_{j-1} ) ]$
\vspace{1cm}

\textbf{Algorithm 2} Delayed Sequential Importance Resampling
\begin{itemize}
\item Sample $X_t^{(i)} \sim q( \cdot | X_{t-1}^{(i)})$
\item $w_T(X_{1:t}^{(i)}) \propto W_T(X_{1:t-1}^{(i)}) \cdot g(X_t^{(i)}, y_t) \cdot
  f(X_t^{(i)} | X_{t-1}^{(i)}) / q(X_t^{(i)} | X_{t-1}^{(i)})$\\
  $w_D(X_{1:t}^{(i)}) \propto W_D(X_{1:t-1}^{(i)}) \cdot g(X_t^{(i)}, y_t) \cdot
  f(X_{t-\delta}^{(i)} | X_{t-1-\delta}^{(i)}) / q(X_{t-\delta}^{(i)} | X_{t-1-\delta}^{(i)})$
\item $W_T(X_{1:t}^{(i)}) = w_T(X_{1:t}^{(i)}) / \sum_{j=1}^N w_T(X_{1:t}^{(j)})$\\
  $W_D(X_{1:t}^{(i)}) = w_D(X_{1:t}^{(i)}) / \sum_{j=1}^N w_D(X_{1:t}^{(j)})$
\item If $ESS_D < N/2$:\\
  resample proportional to $\{W_D(X_{1:t}^{(i)})\}_{i=1,...,N}$\\
  $\forall i$ set $W_T(X_{1:t}^{(i)}) = \frac{W_T(X_{1:t}^{(i)})/W_D(X_{1:t}^{(i)})} {\sum_{j=1}^N W_T(X_{1:t}^{(j)})/W_D(X_{1:t}^{(j)})}$\\
  $\forall i$ set $W_D(X_{1:t}^{(i)}) = 1/N$\\
\end{itemize}

\vspace{1cm}

\noindent Note that for any $t$
$$W_T(X_{1:t}^{(i)}) = W_D(X_{1:t}^{(i)}) \cdot \prod_{k=t-\delta}^{t-1} \frac{ f(X_{k+1}^{(i)} | X_{k}^{(i)}) }{ q(X_{k+1}^{(i)} | X_{k}^{(i)}) }$$
\noindent Define $A_t^{(i)} := \prod_{k=t-\delta}^{t-1} \frac{ f(X_{k+1}^{(i)} | X_{k}^{(i)}) }{ q(X_{k+1}^{(i)} | X_{k}^{(i)}) }$, then 
the resampling step of \textbf{Algorithm 2} becomes
\begin{itemize}
\item If $ESS_D < N/2$:\\
  resample proportional to $\{W_D(X_{1:t}^{(i)})\}_{i=1,...,N}$\\
  $\forall i$ set $W_T(X_{1:t}^{(i)}) = \frac{ A_t^{(i)} } { \sum_{j=1}^N A_t^{(j)} }$\\
  $\forall i$ set $W_D(X_{1:t}^{(i)}) = 1/N$\\
\end{itemize}

\noindent This method is similar to a lookahead strategy where the expected future states
are approximated by one sample. The main distinction is that a lookahead
strategy considers the data ahead of the position where the particles are being resampled.
Our delayed importance sampling looks at the data before the position where the
particles are being resampled. We delay the transition influence rather than looking
ahead at the emission influence. If we were less concerned with the time required
to extend particles at each resampling step and then discard these extensions,
a lookahead strategy would lead to a better approximation. For a fixed computing time
we should consider whether more particles or a more involved rasampling step (lookahead)
would most benefit our approximation. \\

\noindent Now consider the distributions described above in the more specific case of SMC$^2$. 
$f(X_{1:k})$ and $q(X_{1:k})$ are
probability distributions of sequences of trees where the tree index $1,...,k$ corresponds to a nucleotide. 
Both $f()$ and $q()$ are memoryless, so they can be factored into spatial components:
$f(X_{1:k})=f(X_1) \cdot f(X_2|X_1) \cdot f(X_3|X_2) \cdot \ldots \cdot f(X_k|X_{k-1})$.
To simplify notation, consider spatial components where either no recombination has occured or
a single recombination has occured in the final nucleotide position. We will now change our indices
so that $X_t$ is the tree at the beginning of this spatial component and $X_{t+1}$ is the tree
at the end of this spatial component. Now,

\begin{equation}
 f(X_{t+1}|X_{t}) = \begin{cases}
  \exp(- L \cdot B \cdot \rho), & \text{ if } X_{t+1} = X_{t} \\
  B \cdot \rho \cdot \exp(- L \cdot B \cdot \rho), & \text{ otherwise }
 \end{cases}
\end{equation}

\noindent where $L$ is the number of nucleotides that the spatial component $X_t,X_{t+1}$ spans,
B is the total branch length of the tree $X_t$, and $\rho$ is the recombination rate. \\

\noindent We wish to over sample recombination events in specified time sections of the tree space.
To this end, we define out proposal distribution $q()$ such that

\begin{equation}
 q(X_{t+1}|X_{t}) = \begin{cases}
  \exp(- L \cdot B' \cdot \rho), & \text{ if } X_{t+1} = X_{t} \\
  B' \cdot \rho \cdot \exp(- L \cdot B' \cdot \rho), & \text{ otherwise }
 \end{cases}
\end{equation}

\noindent where $B'$ is a rescaled branch length of tree $X_t$. 

\begin{align}
B' = \sum_{i} r_i \cdot B_i 
\end{align}

\noindent In the above equation, $r_i$ is the specified focus ratio for time section $i$, and
$B_i$ is the branch length of $X_t$ which lies in time section $i$. Time sections of interest will be
given a focus ratio $r > 1$. \\

\noindent Both $f()$ and $q()$ can be factored into a spatial and a temporal component.
The spatial component describing the probability of a recombination occuring at the nucleotide 
position of tree $X_{t+1}$. The temporal component describing the probability (given a recombination
occurs at the specified nucleotide position) that the recombination occurs at a particular point in time.

\begin{align}
f_{\text{temporal}}(X_{t+1}|X_{t}) &= 1 \\
f_{\text{spatial}}(X_{t+1}|X_{t}) &= (B \cdot \rho)^{\mathbbm{1}(X_{t+1} \neq X_{t})} \cdot \exp(- L \cdot B \cdot \rho) \\
q_{\text{temporal}}(X_{t+1}|X_{t}) &= (\frac {B'} {r_i \cdot B})^{\mathbbm{1}(X_{t+1} \neq X_{t})} \text{for recombination in time section $i$} \\
q_{\text{spatial}}(X_{t+1}|X_{t}) &= (B' \cdot \rho)^{\mathbbm{1}(X_{t+1} \neq X_{t})} \cdot \exp(- L \cdot B' \cdot \rho)
\end{align}

\noindent Now the importance weight $f()/q()$ can be written as
\begin{align}
\frac{f()}{q()} = \frac{ f_{\text{temporal}}() }{ q_{\text{temporal}}() } \cdot \frac{ f_{\text{spatial}}() }{ q_{\text{spatial}}() }
\end{align}

\noindent We scale the $r_i$'s so the average tree (under our best guess at the demographic model)
has $B=B'$. This results in the number of sampled recombination events approximately equal to 
the number of recombination events expected under the transition. For the average tree under $f(X)$,
$\frac{ f_{\text{spatial}} }{ q_{\text{spatial}} } \approx 1$. Of course this does not hold for all trees.
Still, due to this scaling, when a recombination event is sampled the temporal factor has a 
stronger influence than the spatial factor on the importance weight. 
The goal of delayed importance sampling is to delay the penalty of sampling a recombination event
in one of the up-weighted sections in order to collect evidence in support of the new tree 
before resampling discards the particle based solely on its likelihood under the transition distribution. 
As such, it is sufficient to delay only the temporal factor. \\

\noindent The amount of data needed to provide sufficient support for or against an event in the tree
is dependent on the time of the event (see section on fixed-lag). As such, the delay imposed on a 
recombination event is dependent on the epoch in which that event occurred. We use a delay of
$\text{lag}_{\text{epoch} j}/2$. \\

% wait, temp is stronger influence when there is a recombination, but when there isn't, spatial is the stronger factor...
% hopefully scaling keeps this close to one
% write a few examples...

\noindent Now consider the algorithm for SMC$^2$ with delayed IS

\begin{itemize}
\item Extend particles to next variant
 \begin{itemize}
 \item Update $w_T$ appropriately
 \item Update $w_D$ to include $f_{\text{spatial}}/q_{\text{spatial}}$ and $g$
 \item Store $f_{\text{temporal}}/q_{\text{temporal}}$ in a priority queue
 \item Update $A$ so it reflects everything in the priority queue
 \end{itemize}
\item Additioanl weight management
 \begin{itemize}
 \item Normalize $W_T$
 \item Update $w_D$ to include elements of priority queue with application position less than current variant position and remove these from the queue
 \item Normalize $W_D$
 \end{itemize}
\item Extract events and opportunities weigthed by $W_T$
\item If $ESS_D<N/2$ resample
 \begin{itemize}
 \item Resample particles according to $W_D$
 \item Set $ W_D(X_{1:t}^{(i)}) = 1/N $
 \item Set $ W_T(X_{1:t}^{(i)}) = A_t^{(i)} / \sum_{j=1}^N A_t^{(j)} $
 \end{itemize}
\end{itemize}



\end{document}
